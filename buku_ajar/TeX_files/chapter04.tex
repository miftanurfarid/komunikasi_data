\chapter{Media Transmisi}

Media transmisi adalah jalur fisik antara transmitter dan receiver di sistem transmisi data. Pada bab sebelumnya telah dijelaskan bahwa gelombang elektromagnetik dipandu sepanjang media solid di media terpandu (\textbf{guided media}). Contohnya seperti copper twisted pair, copper coaxial cable dan fiber optik. Sedangkan transmisi nirkabel terjadi melalui atmosfer, luar angkasa atau air di media tak terpandu (\textbf{unguided media})

Karakteristik dan kualitas dari transmisi data ditentukan oleh katakteristik medium dan karakteristik sinyalnya. Pada media terpandu, medium itu sendiri merupakan hal terpenting yang dapat menentukan batasan transmisi. Pada media tak terpandu, bandwidth dari sinyal yang dihasilkan oleh antena transmisi lebih penting dari pada medium dalam penentuan karakteristik transmisi.

Salah satu sifat sinyal yang ditransmisikan melalui antena adalah directionality. Secara umum, sinyal pada frekuensi lebih rendah adalah omni-directional artinya sinyal berpropagasi ke segala arah. Sedangkan pada frekuensi lebih tinggi, sinyal dapat difokuskan menjadi directional beam.

Data rate dan jarak menjadi salah satu pertimbangan dalam desain sistem transmisi data. Tujuannya adalah dicapainya data rate yang tinggi dan jarak jangkauannya yang jauh.  Beberapa faktor dari media transmisi dan sinyal yang menentukan data rate dan jarak jangkauannya adalah sebagai berikut

\begin{itemize}
	\item \textbf{Bandwidth:} Semakin besar bandwidth sinyalnya maka semakin besar data rate yang dapat dicapai.
	\item \textbf{Transmission impairment/ kerusakan transmisi: } Impairments seperti atenuasi dapat membatasi jarak. Pada media terpandu, twisted pair umumnya lebih rentan terhadap impairment daripada coaxial cable dan coaxial cable lebih rentan terhadap impairment daripada fiber optic.
	\item \textbf{Interference:} Interference dari sinyal lain pada overlapping frequency band dapat mendistorsi atau meng-cancel out sinyal tersebut. Umumnya interference terjadi di media tak terpandu, tapi juga terkadang di media terpandu. Pada media terpandu, interference dapat disebabkan alien crosstalk maupun internal crosstalk
\end{itemize}

\section{Media Transmisi Terpandu}

\section{Transmisi Nirkabel}

\section{Propagasi Nirkabel}

\section{Transmisi Line-of-sight}

